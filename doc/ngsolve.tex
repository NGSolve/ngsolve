%
%  latex2html ngsolve.tex -long_titles 3 
%
%

\documentclass[12pt]{book}
\usepackage{a4, epsf}
% \usepackage{html}

\title{NGSolve - 4.4}
\author{Joachim Sch\"oberl}

% \unitlength=1cm

\begin{document}
\maketitle
\tableofcontents

\chapter{Getting Started}

NGSolve is a finite element library. It contains equation assembling
and equation solving.  It does not contain mesh operations, so it has
to be linked to a mesh handler such as Netgen.

\section{Installing NGSolve as Netgen Add-On}

Install Netgen and NGsolve as expained on {\tt http://sourceforge.net/apps/mediawiki/ngsolve}


\section{A first demo}
Start the program 'netgen'. There should pop up
a window with menu items, and a large white visualization area.

Select the menu item {\em Solve $\rightarrow$ Load PDE} and open the PDE file {\em ngsolve/pde\_tutorial/d1\_square.pde}. 
You get some text in the shell window, and you will find a blue square in the drawing area. 
You have loaded the problem description, including geometry and mesh. In the button line (below the menu items) you see the selection-button showing 'Geometry'. Here, you can switch to 'Mesh', and later, you switch to 'Solution'. 
This selects the visualization mode in the drawing area.


Now, you are ready to start the first solve. Press the {\em Solve}-button.. 
The visualization switches to Solution, and you see two gray triangles.

Click {\em Solve$\rightarrow$Visualization} to open the visualization dialog box. Here,
you can select what you want to see. Look for the selection button {\em Scalar Function}, and change from 'None' to 'u'. The triangles will be colored.


Clearly, this mesh is too coarse. Go on, and press the {\em Solve}-button five more times. The mesh gets refined, and new solutions will be 
computed. 


Play around with the items in the visualization box. You may try Autoscale, Isolines, and Textures. Go on and do some more refinement steps (by pressing the short-cuts {\tt <s>-<p>}.


The next chapter will describe the structure of the PDE input file.


\chapter{The PDE Description file}

The way to describe the equations in NGSolve is very close to
the variational formulation (the virtual displacement formulation).

\section{The weak formulation}

The partial differential equation describing a heat flow problem
with thermic conductivity $\lambda$, a volumetric source density $f$, 
an inflow density $g$ at the part $\Gamma_N$ of the boundary, 
and a heat exchange to the environment of temperature $u_0$, and 
conductivity $\alpha$ on $\Gamma_R$ is
\begin{displaymath}
\begin{array}{rcll}
-\mbox {div} \lambda \nabla u & = & f \quad & \mbox{in } \Omega, \\
n^T \lambda \nabla u & = & g  \quad & \mbox{on } \Gamma_N, \\
n^T \lambda \nabla u & = & \alpha (u_0 - u)  \quad & \mbox{on } \Gamma_R.
\end{array}
\end{displaymath}

 
The weak form of the equation is to find the temperature distribution $u$
in the proper Sobolev space $V := H^1(\Omega)$ such that
$$
\int_\Omega \lambda \nabla u \cdot \nabla v dx + \int_{\Gamma_R} \alpha u v \, ds =
\int_\Omega f v \, dx + \int_{\Gamma_N} g v \, ds + \int_{\Gamma_R} \alpha u_0 v \, ds
$$
holds for all test functions $v \in V$. The finite element method replaces the
Sobolev spaces $V$ by a finite dimensional sub-space.

The PDE input file defines the variational problem. You specify the
finite element space, and then, the bilinear-form (left hand side),
and the linear form (right hand side) are build up from elementary
blocks called integrators. The table below lists the names of the
integrators needed in the equation above:

\begin{quote}
\begin{tabular}{|l|l|}
\hline
laplace lam &  $\int_\Omega \lambda \nabla u \cdot \nabla v \, dx$ \\
robin alpha & $\int_{\Gamma_R} \alpha u v \, ds$ \\
\hline
source f    & $\int_\Omega f v \, dx$ \\
neumann g   & $\int_{\Gamma_N} g v \, ds$ \\
\hline
\end{tabular}
\end{quote}

Note that the last term is handled by the neumann - integrator with
coefficient $\alpha u_0$. You might miss Dirichlet boundary
conditions. Indeed, NGSolve always approximates them by Robin
b.c. with large conductivity~$\alpha$.

\section{The input file}


The file below is a valid input file to NGSolve. The first two lines
load the prepared geometry file and the mesh file. It is a square. The
boundary splits into the 4 sides $\Gamma_1$ to $\Gamma_4$. We specify 
Robin boundary conditions (with conductivity $\alpha = 10^5$) on $\Gamma_1$
to $\Gamma_3$, and Neumann boundary condition (with $g = 1$) on $\Gamma_4$.
This is defined in terms of coefficients. The list of numbers correspond to
the sub-domains, if the integral is taken over the domain, or, to parts
of the boundary, if the integral is taken over the boundary, respectively.
The next lines define the mathematical objects finite-element space, 
grid-function, bilinear-form, linear-form, and a preconditioner. 
The last line (numproc) calls the solver for boundary value problems (bvp). Here,
a preconditioned conjugate gradients iteration is called.

\begin{verbatim}
geometry = ngsolve/pde_tutorial/square.in2d
mesh = ngsolve/pde_tutorial/square.vol

define coefficient coef_lam
1, 

define coefficient coef_alpha
1e5, 1e5, 1e5, 0, 

define coefficient coef_f
1,

define coefficient coef_g
0, 0, 0, 1,

define fespace v -order=1
define gridfunction u -fespace=v -nested

define bilinearform a -fespace=v
laplace coef_lam
robin coef_penalty

define linearform f -fespace=v
source coef_source
neumann coef_g

define preconditioner c -type=multigrid -bilinearform=a -smoothingsteps=1

numproc bvp np1 -bilinearform=a -linearform=f -gridfunction=u -preconditioner=c \
    -maxsteps=50
\end{verbatim}



\chapter{Reference Manual}

This is the most up to date description of the PDE input file. 
General rules for writing input files are:
\begin{itemize}
\item
The input file is case sensitive. All keywords must be in lower-case.
\item
Line comments start with the pound sign \#. The rest of the line is ignored.
Block comments are not available.
\item
The input file must start with geometry and mesh definition. The geometry is optional,
the mesh statement is required:
\begin{verbatim}
geometry = geodir/geofilename.geo
mesh = meshdir/meshfilename.vol
\end{verbatim}
The filename is relative to the starting directory of ng. Spaces are optional.
\item
After that, {\tt define} commands and {\tt numproc} commands
follow. Each object must be given a unique name.  The objects are
called in order of definition. The calling action depends on the type
of object. For example, it is memory allocation for grid-functions,
matrix assembling for bilinear-forms, and linear equation solving for
the bvp-numproc.
\item
Many commands require option flags. There are define flags, numerical flags, 
string flags, number-list flags, and string-last flags. Examples for every type are
\begin{verbatim}
-printstatus 
-ref_fac=0.1 -solution=u 
-parameters=[1.0,2,-5] -spaces=[v1,v2,v3]
\end{verbatim}
Flags need an initial '-' character. Spaces within one flag are not allowed.
\end{itemize}

\section{Constants and Variables}
\label{sec_constants}
Constants and variables are defined as follows:
\begin{verbatim}
define constant pi = 3.1415
define constant omega = (2 * pi * 10E3)
define variable t = 0
\end{verbatim}
A constant is given its value at definition. A variable is
initialized to a value, but later, its value may be changed
from a numproc. An example is the time variable for dynamic problems.

The right hand side can be a constant, or a constant expression.
Expressions must be put into brackets. Expressions may contain the following
operations:
\begin{itemize}
\item
$+,-,/,*,( )$
\item
$\sin$, $\cos$, $\tan$, $\exp$, $\log$
\end{itemize}

\section{Coefficient functions}

Coefficient functions are defined as list of constants or functions. 
Each element corresponds to a sub-domain, or a part of the boundary (specified
by the boundary condition number).

\begin{verbatim}
define coefficient coef_lam  10,15, -5, (5*x), (sin(y)),
\end{verbatim}
\begin{itemize}
\item
Functions must be put into brackets. Function may be built as described
in Section~\ref{sec_constants}, but may now additionally contain 
variables. 
\item
Predefined variables are $x$, $y$, and $z$ for the Cartesian coordinates.
\item
Specifying too many coefficients is ok, too less will throw an error exception.
\end{itemize}

Coefficients can be specified by materials. Use the '-material' flag. Currently this version is only supported for CSG geometries.
\begin{verbatim}
define coefficient coef_nu material
        iron 10000
        air  1
        default 1
\end{verbatim}
All subdomains consisting of material 'iron' get the constant coefficient 10000,
the 'air' domain as well as all others get 1.

\section{Finite Element Spaces}
The definition
\begin{verbatim}
define fespace <name> <flaglist>
\end{verbatim}
defines the finite element space {\tt <name>}.
Example:
\begin{verbatim}
define fespace v -order=2 -dim=3
\end{verbatim}
There are various classes of finite element spaces. Default are continuous,
nodal-valued finite element spaces. The following define flags select the type
of spaces 
\begin{quote}
\begin{tabular}{|l|l|}
\hline
non of the flags below & continuous nodal finite element space \\
{\tt -hcurl } & H(curl) finite elements (Nedelec-type, edge elements) \\
{\tt -hdiv  } & H(div) finite elements (Raviart-Thomas, face elements) \\
{\tt -l2    }  & non-continuous elements, element by element \\
{\tt -l2surf } & element by element on surface \\
{\tt -h1ho } & Arbitrary order continuous elements \\
{\tt -hcurlho } & Arbitrary order H(curl) elements \\
{\tt -hdivho } & Arbitrary order H(div) elements \\
{\tt -l2ho } & Arbitrary order non-continuous elements \\
\hline
\end{tabular}
\end{quote}

The following flags specify the finite element spaces
\begin{quote}
\begin{tabular}{|l|l|}
\hline
{\tt -order=<num>} & Order of finite elements \\
{\tt -dim=<num>}   & Number of fields (number of copies of fe), 2 for 2D elasticity \\
{\tt -vec}         & set -dim=spacedim \\
{\tt -tensor}      & set -dim=spacedim*spacedim \\
{\tt -symtensor}   & set -dim=spacedim * (spacedim+1) / 2,  (symmetric stress tensor) \\
{\tt -complex}     & complex valued fe-space \\
\hline
\end{tabular}
\end{quote}

A compound fe-space combines several fe-spaces to a new one. Useful, e.g.,
for Reissner-Mindlin plate models containing the deflection w and
two rotations beta:
\begin{verbatim}
fespace vw -order=2
fespace vbeta -order=1
fespace v -compound -spaces=[vw,vbeta,vbeta]
\end{verbatim}


The fespace maintains the degrees of freedom. On mesh refinement, the space
provides the grid transfer operator (prolongation). High order fe spaces
maintain a lowest-order fespace of the same type for preconditioning.

\subsection{H1-Finite Element Space}

\section{Grid-functions}

A grid-function is a function living in a finite element space. Definition:
\begin{verbatim}
define gridfunction <name> <flaglist>
\end{verbatim}
Example:
\begin{verbatim}
define gridfunction u -fespace=v
\end{verbatim}

The string-flag {\tt fespace} defines the fespace the grid function lives in.
The flag must refer to a valid fespace. 

If the define flag {\tt nested} is specified, the grid-function will be 
prolongated from the coarse space to the fine space when the mesh is
refined.

\section{Bilinear-forms}

A bilinear form is defined by
\begin{verbatim}
define bilinearform <name> <flaglist>
integrator1
integrator2
integrator3
...
\end{verbatim}
Example
\begin{verbatim}
define bilinearform a -fespace=v
laplace lam
robin alpha
\end{verbatim}

A bilinear-form is always defined as sum over integrators. A bilinear-form 
maintains the stiffness matrix. For multi-level algorithms, a bilinear-form 
stores all matrices. Bilinear-forms for high order spaces have a bilinear-form
for the corresponding lowest order space.

The following flags are defined \newline
\begin{tabular}{|l|l|}
\hline
-fespace=<name> & bilinear form is defined on fe space <name> \\
-symmetric    & bilinear form is symmetric (store just lower left triangular matrix)  \\
-nonassemble  & do not allocate matrix (bilinear-form is used, e.g., for post-processing) \\
-project & use Galerkin projection to generate coarse grid matrices \\
\hline
\end{tabular}


An integrator is defined as
\begin{verbatim}
token <coef1> <coef2> ... <flaglist>
\end{verbatim}
Example:
\begin{verbatim}
elasticity coef_e coef_nu -order=4
\end{verbatim}
The <coefi> refers to a coefficient function defined above. It
provides the coefficients defined sub-domain by dub-domain for
integrators defined on the domain (e.g., laplace), or, the coefficient
boundary-patch by boundary-patch for integrators defined on the
surface (e.g., robin).

Allowed flags are \newline
\begin{tabular}{|l|l|}
\hline
-order=num & use integration rule of order num. Default order is computed form element order. \\
-comp=num & use scalar integrator as component num for system (e.g., penalty term for y-displacement). num=0 adds integrator to all components. \\
-normal   & add integrator in normal direction (penalty for normal-displacement) \\
\hline
\end{tabular}

The integrator tokens are \newline
\begin{tabular}{|l|l|}
\hline
laplace lam  &    $\int_\Omega \lambda \nabla u \cdot \nabla v \, dx$ \\
mass rho     &    $\int_\Omega \rho  u  v \, dx$ \\
robin alpha  &    $\int_\Gamma \alpha u v \, ds$ \\
elasticity e nu & $\int_\Omega D \varepsilon(u) : \varepsilon(v) \, dx$  
(with $D$..3D elasticity tensor, or plane stress) \\
\hline
curlcurledge nu &    $\int_\Omega \nu (\nabla \times u) (\nabla \times v) \, dx$ for $H(curl)$ spaces \\
massedge sigma &    $\int_\Omega \sigma u \cdot v dx$ for $H(curl)$ spaces \\
robinedge sigma &   $\int_\Gamma \sigma (n \times u) (n \times v)  ds$ for $H(curl)$ spaces \\
\hline
\end{tabular}







\section{Linear-forms}

A linear form is defined by
\begin{verbatim}
define linearform <name> <flaglist>
integrator1
integrator2
integrator3
...
\end{verbatim}
Example
\begin{verbatim}
define linearform f -fespace=v
source coef_f
neumann coef_g
\end{verbatim}

A linear-form is always defined as sum over integrators. A linear-form 
maintains the right hand side vector.

The following flags are defined \newline
\begin{tabular}{|l|l|}
\hline
-fespace=<name> & bilinear form is defined on fe space <name> \\
\hline
\end{tabular}

An integrator is defined as
\begin{verbatim}
token <coef1> <coef2> ... <flaglist>
\end{verbatim}
Example:
\begin{verbatim}
source coef_fy -comp=2
\end{verbatim}
The <coefi> refers to a coefficient function defined above. It
provides the coefficients defined sub-domain by dub-domain for
integrators defined on the domain (e.g., source), or, the coefficient
boundary-patch by boundary-patch for integrators defined on the
surface (e.g., neumann).

Allowed flags are \newline
\begin{tabular}{|l|l|}
\hline
-order=num & use integration rule of order num. Default order is computed form element order. \\
-comp=num & use scalar integrator as component num for system (e.g., penalty term for y-displacement). num=0 adds integrator to all components. \\
-normal   & add integrator in normal direction (surface load in normal direction) \\
\hline
\end{tabular}

The integrator tokens are \newline
\begin{tabular}{|l|l|}
\hline
source f     &    $\int_\Omega f  v \, dx$ \\
neumann g     &    $\int_\Gamma g  v \, ds$ \\
\hline
sourceedge jx jy jz  &    $\int_\Omega j \cdot v \, dx$  for 3D $H(curl)$ spaces \\
neumannedge jx jy jz  &    $\int_\Gamma (n \times j) \cdot (n \times v) \, ds$  for 3D $H(curl)$ spaces \\
curledge f  &   $\int_\Omega f (\nabla \times v_z) \, dx$  for 2D $H(curl)$ spaces \\
curlboundaryedge f  &    $\int_\Gamma f n \cdot (\nabla \times v) \, ds$  for 3D $H(curl)$ spaces \\
\hline
\end{tabular}






\section{Preconditioners}
A preconditioner is defined by
\begin{verbatim}
define preconditioner <name> -type=<type> <flaglist>
\end{verbatim}
Example:
\begin{verbatim}
define preconditioner c -type=multigrid -bilinearform=a -smoothingsteps=2
\end{verbatim}
A preconditioner must have a type flags specifying the type of preconditioner.
The remaining flags depend on the preconditioner.

Preconditioners are:
\begin{itemize}
\item 
-type=local: Local preconditioner, symmetric Gauss-Seidel Jacobi, block Gauss-Seidel

Flags are
\begin{tabular}{|l|l|}
\hline
-bilinearform=<name> & name of bilinear-form containing matrix \\
-block & use block Gauss-Seidel (block defined by fe-space)
\end{tabular}


\item
-type=multigrid: Multigrid preconditioner

Flags are
\begin{tabular}{|l|l|}
-bilinearform=<name> & name of bilinear-form containing matrix \\
-smoother=<smoother> & type of smoother: 'point'..GS, 'block'..block GS, 'line'..line-GS (lines by anisotropic mesh) \\
-coarsetype=<coarse> & type of coarse grid solver: 'exact'..factorization, 'smoothing'..use smoother, 'cg'..inner cg iteration \\
-smoothingsteps=nsm   & number of pre- and post-smoothing steps \\
-increasesmoothingsteps=inc & smoothing steps on level $l$ are $nsm * inc^{L-l}$ with $L$..finest level\\
-coarsesmoothingsteps=nsmc & smoothing steps for coarse grid solver (if smoother) \\
-finesmoothingsteps=nsmf & smoothing steps for high order equation \\
-test & compute eigenvalues of preconditioned system \\
-timing & measure time per preconditioning step \\
-updateall & update coarse grid matrices (e.g.\ in combination with the bilinearform flag {\it project}) \\
\hline
\end{tabular}

\item -type=direct: Cholesky factorization

\end{itemize}

\section{Numerical Procedures}

Numerical procedures are functions where the actual computations happen.
Numprocs call iterative solvers, perform time integration loops, 
control postprocessing, and error estimators.

\subsection{Boundary Value Problem}

Keyword: {\tt bvp}

The numproc bvp takes a matrix (from a bilinear form), a right hand side
vector (from a gridfunction), call an iterative solver to compute the
solution vector stored in a gridfunction.

Example:
\begin{verbatim}
define bvp np1 -bilinearform=a -linearform=f -solution=u -preconditioner=c
\end{verbatim}

Flags are:
\begin{tabular}{ll}
\hline
-bilinearform=<name> & bilinear form to provide matrix \\
-linearform=<name> & linear form to provide right hand side vector\\
-gridfunction=<name> & gridfunction to store solution \\
-preconditioner=<name> & preconditioner for iterative solver \\
-maxsteps=num & maximal number of iterations \\
-prec=num & relative error reduction \\
-solver={cg|qmr} & choice of iterative solver, default is cg. \\
-print     & print matrix, rhs, and solution to test.out file \\
\hline
\end{tabular}

\subsection{Post processing}

Keyword: {\tt calcflux}

Compute derivatives of solution. Depending on the problem, this function
computes the gradient, the flux, strain or stress, magnetic induction, etc...

Example:
\begin{verbatim}
numproc calcflux pp1 -bilinearform=a -solution=u -flux=p -applyd
\end{verbatim}

Flags are:
\begin{tabular}{ll}
\hline
bilinearform=<name> & The flux is defined by the first integrator of the bilinearform \\
solution=<name>  & The input gridfunction \\
flux=<name> & The output gridfunction. Must be nodal-valued or element by element.\\
-applyd     & apply coefficientmatrix. Switches between stress and strain, or
B-field and H-field \\
\hline
\end{tabular}


\subsection{Evaluation of numerical values}

Keyword: {\tt evaluate}

Evaluate (bi)linear functionals on gridfunctions, compute point values, etc,..

Examples:
\begin{verbatim}
numproc evaluate ev1 -linearform=f -gridfunction=u -text=NormalFlux
numproc evaluate ev2 -bilinearform=a -gridfunction=u -point=[0.3,0.5,0.5] -applyd
\end{verbatim}

Flags are:
\begin{tabular}{ll}
\hline
-bilinearform=<name> & bilinear form to evaluate a(u,v), or, take first
integrator to define quantity to evaluate \\
-linearform=<name>   & linear form to evaluate f(v) \\
-gridfunction=<name> & gridfunction to evaluate \\
-gridfunction2=<name> & gridfunction v to evaluate a(u,v) \\
-point=[px,py,pz] & point where to evaluate $B u$ \\
-point2=[qx,qy,qz] & evaluate elong line [P,Q] and store values in file \\
-filename=<name>   & file to store results \\
-applyd            & switch between stain/stress, i.e., $Bu$ or $DBu$ \\
-text=<text>       & output is ``<text> = `` values \\
\hline
\end{tabular}

\subsection{Error estimator}

Keyword: {\tt zzerrorestimator}

Performs a Zienkiewicz-Zhu type error estimator.

Example:
\begin{verbatim}
numproc zzerrorestimator ee1 -bilinearform=a -solution=u -error=err
\end{verbatim}

Averages the fluxes, and computes difference between averaged flux and
original flux. Difference is added to element by element gridfunction
error. Averaging is done sub-domain by sub-domain to avoid too strong
refinement along material interfaces.

\begin{tabular}{ll}
\hline
-bilinearform=<name> & first integrator of bilinear-form defines flux \\
-solution=<name>   & gridfunction containing solution \\
-error=<name>      & element-by-element, order 0 gridfunction containing element contributions of error \\
\hline
\end{tabular}

\subsection{Refinement marker}

Keyword: {\tt markelements}

Marks elements for mesh refinement.

Example:
\begin{verbatim}
numproc markelements me -error=err -minlevel=2 -fac=0.1
\end{verbatim}

All elements having element error greater than fac times maximal element
error are marked for refinement. 

\begin{tabular}{ll}
\hline
-error=<name> & element by element, order 0, gridfunction containing element contributions \\
-minlevel=<l>  & Do l-1 levels of uniform refinement before starting adaptivity \\
-fac=val   & Elements having contribution greater than val * maximal err are marked \\
-error2=<name> & mark elements based on element by element quantity \\
error$_T$ * error2$_T$, used for goal driven error estimator. \\
\hline
\end{tabular}

\end{document}
